\documentclass[10pt, letterpaper, twocolumn, openany]{book}
\usepackage[utf8]{inputenc}
\usepackage{geometry}
\usepackage{xcolor}
\usepackage{titlesec}
\usepackage{enumitem}
\usepackage{tabularx}
\usepackage{mathpazo} 
\usepackage{ragged2e}
\usepackage{fancyhdr} 
\usepackage{tocloft}
\usepackage{cuted}
\usepackage{multicol}

% --- D&D VISUAL STYLE SETTINGS ---

% Margins
\geometry{top=1.8cm, bottom=1.8cm, left=1.8cm, right=1.8cm}

% Colors
\definecolor{PhbRed}{HTML}{58180D} 

% Chapter Formatting
\titleformat{\chapter}[display]
  {\normalfont\huge\bfseries\color{PhbRed}}
  {\chaptertitlename\ \thechapter}
  {20pt}
  {\Huge\scshape}

% Section Formatting
\titleformat{\section}
  {\normalfont\Large\scshape\color{PhbRed}}
  {}{0em}{}
  [\titlerule]

\titleformat{\subsection}
  {\normalfont\large\bfseries\color{PhbRed}}
  {}{0em}{}

% Table Styling
\renewcommand{\arraystretch}{1.3}

% Remove paragraph indentation
\setlength{\parindent}{0pt}
\setlength{\parskip}{0.5em}

% Header/Footer
\pagestyle{fancy}
\fancyhf{}
\fancyfoot[C]{\thepage}
\renewcommand{\headrulewidth}{0pt}

% --- STAT BLOCK & SIDEBAR SETTINGS ---
\usepackage[most]{tcolorbox}
\tcbuselibrary{breakable} % <--- 1. ENABLE BREAKING LIBRARY

% Define the Colors
\definecolor{StatBg}{HTML}{FDF1DC}
\definecolor{StatFrame}{HTML}{E69A28}
\definecolor{SidebarBg}{HTML}{E0E5C1}
\definecolor{PhbRed}{HTML}{58180D}

% Define the Monster Stat Block Box
\newtcolorbox{monsterbox}[1]{
    colback=StatBg,
    colframe=StatFrame,
    fonttitle=\bfseries\scshape\Large\color{PhbRed},
    coltitle=PhbRed,
    title=#1,
    sharp corners,
    boxrule=1.5pt,
    enhanced,
    /tcb/size=small
}

% Define the Green Sidebar Box
\newtcolorbox{sidebarbox}[1]{
    colback=SidebarBg,
    colframe=SidebarBg,
    fonttitle=\bfseries\scshape\large\color{black},
    coltitle=black,
    title=#1,
    sharp corners,
    boxrule=0pt,
    enhanced,
    /tcb/size=small
}

\newcommand{\statrule}{\vspace{2pt}\hrule height 1pt \color{PhbRed}\vspace{4pt}}

% --- DOCUMENT CONTENT ---

\begin{document}

% =========================================================
% TITLE PAGE
% =========================================================
\begin{titlepage}
    \centering
    \vspace*{2cm}
    
    {\Huge \scshape \color{PhbRed} Trick's Ledger of \\ Many Tricks \par}
    \vspace{1cm}
    {\Large \textit{A Compendium of Debts, Wagers, and Wildcards} \par}
    \vspace{4cm}
    
    \textbf{Author:} Trick the Trickster \\
    \textbf{Version:} 0.0.6 \\
    
    \vfill
    
    {\small ``Life is not a gift. It is a loan. And I am here to collect.'' \\ --- The Dealer}
\end{titlepage}

% =========================================================
% TABLE OF CONTENTS
% =========================================================
\tableofcontents
\newpage

% =========================================================
% INTRODUCTION
% =========================================================
\chapter*{Introduction}
\addcontentsline{toc}{chapter}{Introduction}

\begin{quote}
\textit{``You say you want a fair game? My dear friend, fairness is a statistical anomaly. I offer you something much better: I offer you a chance to win. Just... mind the fine print.''}
\begin{flushright}
--- The Dealer
\end{flushright}
\end{quote}

\vspace{0.5cm}

\textbf{Trick's Ledger of Many Tricks} is a compilation of options for those who find the standard rules of existence too... restrictive. It is built upon the lore of an entity known variously as the Golden Hand, the Keeper of the Loophole, or simply \textbf{The Trickster}.

\section*{The Two Faces}
To mortals, this entity presents himself as \textbf{The Dealer}. He appears in moments of dire desperation—a warrior bleeding out on the battlefield, a wizard whose spell has failed, a thief cornered in an alley.

The Dealer wears a mask of perfect porcelain neutrality. He offers a simple transaction: power, life, or luck, in exchange for a signature in his Ledger. He speaks of ``The Rules'' with religious reverence. He claims that the universe is a casino, and he is merely the floor manager ensuring the game is played correctly.

\subsection*{The House Edge}
However, the Dealer is a lie. Beneath the mask is the Trickster, an ancient spirit of chaos who despises destiny. He does not play fair. Every contract in this Ledger contains a trapdoor. Every magical item has a cost hidden in the mechanics. Every boon creates a vulnerability.

The Trickster does not cheat by breaking the laws of the universe; he cheats by bending them until they snap. When a character chooses an option from this book, they are sitting down at a table where the cards are marked, the dice are loaded, and the House always wins.

\section*{Using This Book}
This supplement provides new \textbf{Subclasses}, \textbf{Lineages}, \textbf{Feats}, \textbf{Supernatural Gifts}, and \textbf{Magic Items}. As a Dungeon Master, you play the role of the Dealer. You offer these powers to your players. Be generous. Be fair. Let them feel powerful. And then, when the moment is right... reveal the trick.

\newpage

% =========================================================
% CHAPTER 1: SUBCLASSES
% =========================================================

\chapter{Subclasses}
The multiverse is a place of infinite possibility, where luck and skill often blur into one. These archetypes represent those who have mastered the art of the gamble.

% =======================================================
% PALADIN SECTION 
% =======================================================

\section{Paladin Sacred Oaths}
At 3rd level, a paladin gains the Sacred Oath feature. The following options are available to paladins in addition to those in the \textit{Player's Handbook}: the Oath of the Ledger and the Oath of the Debtor.

% -------------------------------------------------------
% OATH OF THE LEDGER
% -------------------------------------------------------
\subsection{Oath of the Ledger}
\textit{"The House always wins. My hammer is just the receipt."}

Paladins who swear the Oath of the Ledger are the cosmic debt collectors of the multiverse. They don't care about good or evil—they care about balance, contracts, and the settle-up. They hunt oathbreakers, tax evaders of the soul, and those who try to cheat their destiny.

Their armor is often etched with golden numbers that shift on their own, and their smites ring with the sound of dropping coins.

\subsubsection{Tenets of the Ledger}
The tenets of the Oath of the Ledger are written in invisible ink that burns when read.
\begin{itemize}
    \item \textbf{Honor the Contract.} A deal is a deal. No fine print can save you from the spirit of the agreement.
    \item \textbf{Collect What is Owed.} Whether it is gold, blood, or a soul, you must ensure payment is made.
    \item \textbf{Zero Sum.} For every stroke of luck, there is a cost. You are that cost.
\end{itemize}

\subsubsection{Oath Spells}
You gain oath spells at the paladin levels listed.

\begin{center}
\begin{tabularx}{\linewidth}{l X}
\textbf{Paladin Level} & \textbf{Spells} \\
3rd & \textit{hellish rebuke, hunter's mark} \\
5th & \textit{hold person, zone of truth} \\
9th & \textit{bestow curse, slow} \\
13th & \textit{locate creature, staggering smite} \\
17th & \textit{dominate person, geas} \\
\end{tabularx}
\end{center}

\subsubsection{Channel Divinity}
When you take this oath at 3rd level, you gain the following two Channel Divinity options.

\textbf{Asset Seizure.} As an action, you present your holy symbol and demand payment from a creature within 30 feet that you can see. The creature must succeed on a Charisma saving throw. On a failed save, one object the creature is holding (such as a weapon or arcane focus) flies from its grasp and into your open hand (or lands at your feet if your hands are full). If the object is magical, the creature takes force damage equal to your paladin level.

\textbf{Garnish Wages.} As a reaction when a creature hits you with an attack, you can force them to pay the price. The attacker takes force damage equal to the damage they just dealt to you. This damage ignores resistance.

\subsubsection{Aura of Interest}
Starting at 7th level, you emit an aura that taxes those who harm your allies. Whenever a hostile creature within 10 feet of you deals damage to you or a friendly creature, the attacker takes psychic damage equal to your Charisma modifier (minimum of 1).

At 18th level, the range of this aura increases to 30 feet.

\subsubsection{Predatory Lending}
Starting at 15th level, you can advance credit to your allies at a steep cost to your enemies. When a creature you can see within 30 feet of you drops to 0 hit points, you can use your reaction to grant a friendly creature within 30 feet temporary hit points equal to half the damage that reduced the target to 0.

\subsubsection{The Repossessor}
At 20th level, you can assume the form of the Ultimate Enforcer. Golden chains float around you, and your eyes burn like molten gold. As an action, you undergo a transformation that lasts for 1 minute:
\begin{itemize}
    \item You gain a flying speed of 60 feet (hover).
    \item You have truesight out to a range of 120 feet.
    \item When you hit a creature with a weapon attack, you can force it to make a Charisma saving throw. On a failure, the creature is \textbf{banished} (as per the spell) to a solitary confinement demiplane until the start of your next turn.
\end{itemize}
Once you use this feature, you can't use it again until you finish a long rest.

\vspace{0.5cm}

% -------------------------------------------------------
% OATH OF THE DEBTOR
% -------------------------------------------------------
\subsection{Oath of the Debtor}
\textit{"I don't fight for glory. I fight because if I stop, the bill comes due."}

The Oath of the Debtor is rarely sworn willingly. It is the path of those who have gambled their lives and lost, only to be given a second chance at a terrible interest rate. These paladins are bound to the Proprietor or similar cosmic entities, serving as living collateral.

Their magic feels cold and desperate, often manifesting as spectral chains, coins that rust upon touch, or shadows that bleed.

\subsubsection{Tenets of the Debtor}
A Debtor's vows are less a code of honor and more a repayment plan.
\begin{itemize}
    \item \textbf{Buy Time.} You are living on borrowed time. Make every second count.
    \item \textbf{Pass the Buck.} If someone has to pay the price, ensure it isn't you.
    \item \textbf{Never Default.} The consequences of breaking this contract are worse than death.
\end{itemize}

\subsubsection{Oath Spells}
You gain oath spells at the paladin levels listed.

\begin{center}
\begin{tabularx}{\linewidth}{l X}
\textbf{Paladin Level} & \textbf{Spells} \\
3rd & \textit{false life, hex} \\
5th & \textit{blur, darkness} \\
9th & \textit{life transference, vampiric touch} \\
13th & \textit{death ward, shadow of moil} \\
17th & \textit{enervation, reincarnate} \\
\end{tabularx}
\end{center}

\subsubsection{Channel Divinity}
When you take this oath at 3rd level, you gain the following two Channel Divinity options.

\textbf{Borrow Against the Future.} As a bonus action, you immediately gain temporary hit points equal to twice your paladin level + your Charisma modifier. These temporary hit points last for 1 minute. When the effect ends (or if you lose all the temporary hit points), you take necrotic damage equal to the amount of temporary hit points you originally gained. This damage cannot be reduced in any way.

\textbf{Transfer Liability.} As a reaction when you take damage from a creature within 30 feet of you, you can force that creature to make a Charisma saving throw. On a failed save, the creature takes the damage instead of you. On a successful save, you both take half the damage.

\subsubsection{Aura of Collateral}
Starting at 7th level, you project an aura of shared risk. The aura extends 10 feet from you in every direction.

Whenever you take damage, you can choose to deal necrotic damage equal to your Charisma modifier to a hostile creature within the aura.

At 18th level, the range of this aura increases to 30 feet.

\subsubsection{Compound Interest}
Starting at 15th level, your desperation fuels your power. When you start your turn with less than half of your maximum hit points, you regain hit points equal to your Charisma modifier, and your melee weapon attacks deal an extra 1d8 necrotic damage until the end of the turn.

\subsubsection{Chapter 11}
At 20th level, you can declare spiritual bankruptcy to reset the board. As an action, you become an Avatar of Ruin for 1 minute:
\begin{itemize}
    \item You are immune to necrotic and poison damage.
    \item If you would be reduced to 0 hit points, you instead drop to 1 hit point and can immediately teleport up to 30 feet to an unoccupied space. You can use this effect once per round.
    \item Any creature that starts its turn within 30 feet of you has disadvantage on all saving throws.
\end{itemize}
Once you use this feature, you can't use it again until you finish a long rest.

% ---------------------------------------------------------
% WARLOCK
% ---------------------------------------------------------
\section{Warlock: The Proprietor}
\textit{``The House does not gamble. The House invests. And you, my dear, are a very high-risk asset.''}

You have signed a contract of employment with the Proprietor. Your job is to enforce the House Rules, collect debts, and ensure that the odds always favor your patron.

\subsection*{Expanded Spell List}
The Proprietor lets you choose from an expanded list of spells when you learn a warlock spell. Most involve control, surveillance, or enforcement of terms.

\begin{center}
\renewcommand{\arraystretch}{1.5}
\begin{tabularx}{\linewidth}{c X}
    \hline
    \textbf{Spell Level} & \textbf{Spells} \\
    \hline
    1st & \textit{command, identify} \\
    2nd & \textit{detect thoughts, zone of truth} \\
    3rd & \textit{clairvoyance, glyph of warding} \\
    4th & \textit{locate creature, private sanctum} \\
    5th & \textit{dominate person, geas} \\
    \hline
\end{tabularx}
\end{center}

\subsection*{The House Edge}
\textit{``The odds are never in your favor.''}

Starting at 1st level, you learn to manipulate probability in your favor. When a creature within 60 feet of you makes an attack roll, ability check, or saving throw with advantage, you can use your reaction to cancel the advantage. If you do so, you gain a \textbf{House Chip} (a d6). Within the next minute, you can roll this chip and add the number rolled to one attack roll, ability check, or saving throw you make.

You can use this feature a number of times equal to your proficiency bonus, and you regain all expended uses when you finish a long rest.

\subsection*{VIP Substitution}
\textit{``My best employees get... special protections.''}

At 6th level, your patron ensures their assets are protected. When you are hit by an attack, you can use your reaction to swap places with a willing creature within 30 feet of you. If you have cast \textit{hex} on a creature within 30 feet, you can swap places with them (unwillingly) instead. The creature you swapped with takes the hit.

Once you use this feature, you can't use it again until you finish a short or long rest.

\subsection*{Eye in the Sky}
\textit{``I see everything. Even the cards you haven't drawn yet.''}

At 10th level, your senses are heightened to detect cheats and frauds. You are immune to being charmed. You gain proficiency in the Insight skill. If you are already proficient, you gain expertise. Additionally, you gain \textbf{truesight} out to a range of 15 feet.

\subsection*{Foreclosure}
\textit{``Your debt is due. Immediately.''}

At 14th level, you can call in a debt with lethal force. As an action, you touch a creature. The target must make a Charisma saving throw against your warlock spell save DC. On a failed save, the target takes \textbf{55 (10d10) force damage}. On a successful save, it takes half as much damage. On a successful save, it takes half as much damage.

If this damage reduces the target to 0 hit points, you rip a fragment of their soul free, manifesting as a \textbf{Soul Coin}. You can consume this coin as a bonus action to regain one warlock spell slot.

\subsection*{The House Always Wins}
\textit{``Did you really think you could beat the Dealer?''}

At 20th level, you can channel the ultimate authority of the Proprietor. As an action, you can transform for 1 minute. While transformed:
\begin{itemize}
    \item You are immune to critical hits.
    \item When a creature you can see makes a d20 roll, you can use your reaction to treat the roll as a \textbf{10}. You can do this a number of times equal to your Charisma modifier.
    \item You can cast \textit{counterspell} or \textit{dispel magic} at 5th level without expending a spell slot.
\end{itemize}
Once you use this feature, you can't use it again until you finish a long rest.

% FORCE NEW PAGE FOR CLERIC
\clearpage

% ---------------------------------------------------------
% CLERIC
% ---------------------------------------------------------
\section{Cleric: Chance Domain}
\textit{``Certainty is a cage. Faith is the courage to roll the dice when you can't see the numbers.''}

Clerics of Chance believe that the only truth in the universe is the Golden Roll. They reject destiny and embrace the chaos of the moment.

\subsection*{Domain Spells}
\begin{center}
\renewcommand{\arraystretch}{1.5}
\begin{tabularx}{\linewidth}{c X}
    \hline
    \textbf{Cleric Level} & \textbf{Spells} \\
    \hline
    1st & \textit{chaos bolt, bane} \\
    3rd & \textit{mirror image, nystul's magic aura} \\
    5th & \textit{blink, bestow curse} \\
    7th & \textit{confusion, death ward} \\
    9th & \textit{mislead, skill empowerment} \\
    \hline
\end{tabularx}
\end{center}

\subsection*{Gambler's Instinct}
\textit{``I like those odds.''}

At 1st level, you gain proficiency with gaming sets and the Deception skill. You learn the \textit{guidance} cantrip if you don't already know it. When you cast \textit{guidance}, the target rolls \textbf{2d4} and chooses the highest number to add to the check.

\subsection*{Cascading Fortune}
\textit{``Fortune favors the bold, and hits them twice.''}

Also at 1st level, whenever you roll a die for damage or healing for a spell of 1st level or higher, if you roll the maximum possible number on the die, you can roll that die again and add the new result to the total. This can happen multiple times if you continue to roll the maximum.

\subsection*{Channel Divinity: Twist of Fate}
\textit{``Let's try that again, shall we?''}

At 2nd level, you can use your Channel Divinity to alter a pivotal moment. As a reaction when a creature within 30 feet of you makes an attack roll, ability check, or saving throw, you can force the creature to reroll the d20. You make this choice after seeing the roll but before the DM says whether the roll succeeds or fails.

\subsection*{50/50 Shot}
\textit{``Heads I live, tails... well, let's hope for heads.''}

At 6th level, you can gamble with your own safety. When you take damage, you can use your reaction to roll a \textbf{d6}.
\begin{itemize}
    \item \textbf{1-3:} You take the damage as normal.
    \item \textbf{4-6:} You take \textbf{0 damage} and immediately teleport up to 10 feet to an unoccupied space.
\end{itemize}
You can use this feature a number of times equal to your Wisdom modifier (minimum of once). You regain expended uses when you finish a long rest.

\subsection*{Potent Spellcasting}
\textit{``The House always gets its cut... and so do I.''}

Starting at 8th level, you add your Wisdom modifier to the damage you deal with any cleric cantrip.

\subsection*{The Perfect Run}
\textit{``The Golden Roll. It happens once in a lifetime.''}

At 17th level, you master the flow of chance.
\begin{itemize}
    \item If you roll a 1 on a d20, it counts as a \textbf{20}.
    \item Your \textbf{Cascading Fortune} feature now triggers if you roll the highest \textbf{two numbers} on a damage or healing die (e.g., rolling a 5 or 6 on a d6).
\end{itemize}

% FORCE NEW PAGE FOR ROGUE
\clearpage

% ---------------------------------------------------------
% ROGUE
% ---------------------------------------------------------
\section{Rogue: The Cardsharp}
\textit{``A deck of cards is 52 opportunities for a betrayal. I only need one.''}

You blend arcane infusion with sleight of hand, turning playing cards into deadly weapons.

\subsection*{Deadly Dealing}
\textit{``Pick a card. Any card. It'll be the last thing you see.''}

At 3rd level, you gain proficiency with playing cards. You treat playing cards as simple ranged weapons with the finesse and thrown properties (range 30/60). On a hit, they deal \textbf{1d6 slashing damage}. These cards count as magical for the purpose of overcoming resistance.

\subsection*{The Draw}
\textit{``Hearts for burning passion, Spades for the grave.''}

At 3rd level, your Sneak Attack is infused with the magic of the suits. When you deal Sneak Attack damage with a playing card, roll a \textbf{d4} to determine a bonus effect:
\begin{itemize}
    \item \textbf{1 (Hearts):} Deal extra fire damage equal to your Dexterity modifier.
    \item \textbf{2 (Clubs):} The target must succeed on a Strength saving throw (DC = 8 + Prof + Cha) or be pushed 10 feet away.
    \item \textbf{3 (Diamonds):} Deal extra force damage equal to your Dexterity modifier.
    \item \textbf{4 (Spades):} The target cannot regain hit points until the start of your next turn.
\end{itemize}

\subsection*{Fold 'Em}
\textit{``You can't read a face that isn't there.''}

At 9th level, you become impossible to read. You are immune to magic that allows other creatures to read your thoughts, determine whether you are lying, or know your alignment. You can also perfectly palm Small or smaller objects, making them impossible to detect on your person without a physical search.

\subsection*{The Forced Shuffle}
\textit{``I think it's time we switched seats.''}

At 13th level, you can manipulate positioning on the battlefield. As a bonus action, you can throw a card at a creature within 30 feet. The target must make a Charisma saving throw (DC = 8 + Prof + Cha).
\begin{itemize}
    \item \textbf{Failure:} You and the target instantly swap places.
    \item \textbf{Success:} You instantly teleport to an unoccupied space within 5 feet of the target.
\end{itemize}
You can use this feature a number of times equal to your Charisma modifier (minimum of once). You regain expended uses on a long rest.

\subsection*{52-Card Pickup}
\textit{``Let's see how you handle a bad beat.''}

At 17th level, you can unleash your entire deck in a storm of blades. As an action, you spray cards in a 30-foot cone. Each creature in that area must make a Dexterity saving throw (DC = 8 + Prof + Dex).
\begin{itemize}
    \item \textbf{Failure:} \textbf{8d6 slashing damage} + \textbf{8d6 force damage}.
    \item \textbf{Success:} Half damage.
\end{itemize}
You can choose one creature who failed the save to apply your Sneak Attack damage to. Once you use this feature, you cannot use it again until you finish a short or long rest.

% =========================================================
% CHAPTER 2: LINEAGES
% =========================================================
\chapter{Lineages}
Some are born into the game; others are dragged into it. These lineages represent those whose very blood has been altered by luck, fate, or a deal gone wrong.

\section{Chanceborn}
\textit{``I didn't mean to do that. But I'm glad I did.''}

You were born during a convergence of improbable events—a coin landing on its edge, a solar eclipse at midnight, or a dice roll that defied physics. The Trickster's influence runs in your veins, making you a living anomaly in the laws of probability.

\subsection*{Born of Chaos}
Chanceborn look mostly like their mortal parents, but they always bear the mark of the odds. This might manifest as \textbf{pupils shaped like the front projection of a twenty-sided die}, a shadow that moves slightly out of sync with their body, or the faint, rhythmic sound of tumbling dice whenever they walk. They are often restless, driven by an instinctive need to push their luck.

\subsection*{Living Talismans}
Society views Chanceborn with a mix of awe and superstition. In some cultures, they are considered living good luck charms, invited to weddings and business deals to bless the proceedings. In others, they are shunned as agents of chaos who bring instability wherever they go. A Chanceborn rarely stays in one place for long; they drift wherever the winds of fate—or the flip of a coin—take them.

\subsection*{Chanceborn Names}
Chanceborn are often given names related to luck, gambling, or sudden events.

\textbf{Names:} Ace, Chance, Cipher, Dice, Echo, Fate, Felix, Hazard, Jinx, Kismet, Lucky, Merit, Omen, Penny, Rook, Seven, Venture.

\subsection*{Chanceborn Traits}
\begin{itemize}
    \item \textbf{Ability Score Increase:} Your \textbf{Wisdom} score increases by 2, and one other ability score of your choice increases by 1.
    \item \textbf{Creature Type:} You are a Humanoid.
    \item \textbf{Size:} You are Medium or Small.
    \item \textbf{Speed:} Your walking speed is 30 feet.
    \item \textbf{Innate Luck:} When you roll a \textbf{Natural 1} on the d20 for an attack roll, an ability check, or a saving throw, you can reroll the die and must use the new roll.
    \item \textbf{Serendipity:} You have proficiency in the \textbf{Insight} or \textbf{Perception} skill (your choice).
    \item \textbf{Twist of Fate:} When you make an attack roll, an ability check, or a saving throw and dislike the result, you can give yourself \textbf{Advantage} on that roll. You can use this ability after you roll, but before the DM determines the outcome. Once you use this trait, you can't use it again until you finish a long rest.
\end{itemize}

% =========================================================
% CHAPTER 3: FEATS
% =========================================================
\chapter{Feats}
These new feats represent the specialized tricks, hustles, and gambits learned by those who live on the edge of luck.

\section*{Card Shark}
\textit{Prerequisite: Dexterity 13 or higher}

\textit{``It's not gambling if you know you're going to win.''}

You have mastered the art of throwing cards with deadly precision. You gain the following benefits:
\begin{itemize}
    \item Increase your Dexterity score by 1, to a maximum of 20.
    \item You gain proficiency with playing cards. If you are already proficient with them, you add double your proficiency bonus to checks you make with them.
    \item You can treat ordinary playing cards as simple ranged weapons with the finesse and thrown properties (range 30/60). On a hit, they deal \textbf{1d6 slashing damage}. If you are already proficient with playing cards as weapons (such as from the Cardsharp archetype), the damage die increases by one step (to \textbf{1d8}).
    \item When you take the Attack action to throw a playing card, you can use a bonus action to throw another card.
\end{itemize}

\section*{Cheat Death}
\textit{``The grave can wait. I have one last hand to play.''}

You have a standing arrangement with the powers that govern death—or perhaps you just know how to bluff the Reaper. You gain the following benefits:
\begin{itemize}
    \item Increase your Constitution or Charisma score by 1, to a maximum of 20.
    \item When you are reduced to 0 hit points but not killed outright, you can drop to 1 hit point instead. You can't use this feature again until you finish a long rest.
    \item You have \textbf{advantage} on death saving throws.
\end{itemize}

% =========================================================
% CHAPTER 4: SUPERNATURAL GIFTS
% =========================================================

\chapter{Supernatural Gifts}
Not all power comes from study or devotion. Some power is bought, bargained for, or stolen.

\section{Brand of the Dead Man's Wager}
\textit{Supernatural Gift (mark)}

\textit{``The ink never dries on a contract with the Proprietor. It just burns deeper.''}

This brand appears on your skin as a searing mark of your debt to the Trickster. It manifests as a burning sun with 8 rays, surrounded by 20 iron chain links that seem to tighten as time passes. While this mark is on your person, you feel a constant, feverish heat during the day and a phantom weight dragging at your soul at night.

\subsection*{The Metabolic Clock}
The mark binds your vitality to the cycle of the sun and the debt you owe.
\begin{itemize}
    \item \textbf{Active Phase (Hours 1-8):} You gain a \textbf{+2 bonus to AC}. The sun on the brand glows brightly, but one ray extinguishes every hour.
    \item \textbf{Burden Phase (Hours 9-20):} The sun goes dark. You lose the AC bonus. You cannot benefit from a Long Rest until the 20th hour of the cycle passes.
\end{itemize}

\subsection*{Rolling the Bones}
During the Burden Phase, the Dealer comes to collect interest on your borrowed time. At the end of every hour, roll a \textbf{d20}. 
\textbf{On a roll of 1}, you trigger a Corruption Effect. Roll a \textbf{d8} on the table below:

\begin{center}
\renewcommand{\arraystretch}{1.5}
\begin{tabularx}{\linewidth}{c X} 
    \hline
    \textbf{d8} & \textbf{Effect} \\
    \hline
    1 & \textbf{Singed Nerves:} You take 1d4 fire damage that cannot be reduced. \\
    2 & \textbf{Phantom Weight:} Your speed is reduced by 10 feet for 1 hour. \\
    3 & \textbf{Creaking Joints:} Disadvantage on Stealth checks for 1 hour. \\
    4 & \textbf{Spectral Whispers:} Disadvantage on Perception checks for 1 hour. \\
    5 & \textbf{Lead Arm:} Disadvantage on Strength checks and saving throws for 1 hour. \\
    6 & \textbf{Metabolic Tax:} Your hit point maximum is reduced by \textbf{1d6 + your level} until you finish a Long Rest. \\
    7 & \textbf{Feverish Haze:} Disadvantage on Intelligence and Charisma checks for 1 hour. \\
    8 & \textbf{Grace Period:} The cooldown ends immediately. You may begin a Long Rest. \\
    \hline
\end{tabularx}
\end{center}

% =========================================================
% CHAPTER 5: MAGIC ITEMS
% =========================================================

\chapter{Magic Items}
The tools of the Trickster are rarely straightforward weapons of war. They are puzzles, wagers, and instruments of chaos scattered across the multiverse to tempt the bold and the desperate.

\section{Coin of the Double-Cross}
\textit{Wondrous Item, Uncommon (Requires Attunement)}

\textit{``Fairness is a lie told by the loser. I prefer to win.'' --- Trick the Trickster}

This heavy gold coin features the smiling mask of the Trickster on one side (``Heads'') and the weeping mask of the Dealer on the other (``Tails''). While attuned to it, the coin always feels warm to your touch.

\subsection*{The Fix}
You can manipulate the outcome of the toss to suit your needs. When you flip this coin, you can mentally command it to land on \textbf{Heads} or \textbf{Tails}. You can use this to settle disputes, cheat at gambling, or fool NPCs into thinking Fate has made a decision.

\subsection*{Heads I Win, Tails You Lose}
As a \textbf{bonus action}, you can flick the coin into the air to distract a creature you can see within 30 feet of you. The coin flashes with hypnotic light, forcing the creature to make a \textbf{DC 13 Wisdom saving throw}.

If they fail, their attention is fixated on the coin, triggering an effect based on how you mentally commanded it to land:
\begin{itemize}
    \item \textbf{Heads (The Smile):} You gain \textbf{Advantage} on the next attack roll you make against that creature before the end of your turn.
    \item \textbf{Tails (The Weep):} The creature has \textbf{Disadvantage} on the next attack roll it makes against you before the start of your next turn.
\end{itemize}
The coin magically returns to your pocket at the start of your next turn.

\subsection*{The Edge (Curse)}
If you use the combat feature and roll a \textbf{Natural 1} on your attack (Heads) or the enemy rolls a \textbf{Natural 20} on their attack (Tails), the coin lands on its Edge. It vanishes instantly, returning to the Trickster's vault.

\vspace{0.5cm}
\hrule
\vspace{0.5cm}

\section{Deck of Empty Fates}
\textit{Weapon (Deadly Playing Card), Rare (Requires Attunement)}

\textit{``The future isn't written in the stars. It's written in the hand you're dealt... and I'm the one shuffling.''}

To the naked eye, this appears to be a deck of pristine, white vellum cards with no markings. They feel cool to the touch, like polished marble. The ink only appears the moment the card strikes a living target, writing their fate in real-time.

\subsection*{The Infinite Deal}
You gain a \textbf{+1 bonus} to attack and damage rolls made with this magic weapon. The deck is never depleted. Immediately after you make a ranged attack with a card, it vanishes and reappears in the deck. It functions as a Deadly Playing Card (Finesse, Thrown 30/60).

\subsection*{Fate Revealed}
The cards remain blank on normal hits. However, when you score a \textbf{Critical Hit} (Natural 20) with this weapon, the card flashes with magical ink, revealing a Face Card. Roll a \textbf{d4}:
\begin{enumerate}
    \item \textbf{Jack of Knaves:} You regain hit points equal to the damage dealt.
    \item \textbf{Queen of Sorrows:} The target is \textbf{Incapacitated} until the start of your next turn.
    \item \textbf{King of Ruin:} The target is \textbf{Frightened} of you until the end of your next turn.
    \item \textbf{Ace of Endings:} Instead of doubling the damage dice, you \textbf{triple} the damage dice.
\end{enumerate}

\subsection*{The Joker}
If you roll a \textbf{Natural 1} on an attack roll with this deck, you draw the Joker. The card explodes in your hand. You take \textbf{2d6 force damage}, and you cannot use the Deck again until the start of your next turn.

% =========================================================
% CHAPTER 6: NEW WEAPONS
% =========================================================

\newpage

\chapter{New Weapons}
The Trickster's arsenal includes weapons that are not found in a typical armory. These specialized tools require a deft hand and a deceptive mind to master.

\section*{Weapon Descriptions}
The following new weapon is available to characters who have proficiency with it, most notably the Cardsharp rogue archetype.

\subsection*{Weapon Properties}
\textbf{Deadly Playing Card.} \textit{``Sharp as a razor, light as a feather, and deadly as sin.''} These cards are magically infused or razor-edged, designed to slice through air and armor alike.

\begin{itemize}
    \item \textbf{Finesse:} When making an attack with a finesse weapon, you use your choice of your Strength or Dexterity modifier for the attack and damage rolls. You must use the same modifier for both rolls.
    \item \textbf{Thrown:} If a weapon has the thrown property, you can throw the weapon to make a ranged attack. If the weapon is a melee weapon, you use the same ability modifier for that attack roll and damage roll that you would use for a melee attack with the weapon.
\end{itemize}

% Table automatically handled by float environment
\begin{table*}[t]
\caption{Weapons}
\centering
\renewcommand{\arraystretch}{1.5}
\begin{tabularx}{\textwidth}{l l l l X}
    \hline
    \textbf{Name} & \textbf{Cost} & \textbf{Damage} & \textbf{Weight} & \textbf{Properties} \\
    \hline
    \multicolumn{5}{l}{\textit{\textbf{Simple Ranged Weapons}}} \\
    Deadly Playing Card & --- & 1d6 slashing & --- & Finesse, Thrown (range 30/60) \\
    \hline
\end{tabularx}
\end{table*}

% =========================================================
% CHAPTER 7: MONSTERS
% =========================================================

\chapter{Monsters}
The darker corners of the Proprietor's ledger are not filled with gold, but with things that should not exist.

\section*{Unluck Eater}
\textit{"It looked at me with those big, watering eyes, and I swear I felt my sword hand twitch. It wasn't fear. It was the feeling of missing a parry that hadn't happened yet."}
\\—Sergeant Darrow, shortly before the incident at the Gilded Rose.

To the untrained eye, an Unluck Eater appears to be a mundane, albeit slightly uncanny, forest spirit or exotic pet—resembling a spherical, fluffy mammal with overly large ears and eyes that reflect a little too much light. It behaves with the affection of a house cat, purring loudly when its companions make mistakes.

However, those who study the arcane know them as \textbf{Probability Parasites}. They do not originate from the natural world but leak into the Material Plane from the gaps between the Feywild and the Far Realm.

\subsection*{The Nature of the Beast}
\textbf{The Purr of Entropy.} The Unluck Eater does not verbalize like a normal beast. Its "purr" is actually a low-frequency vibration that sounds remarkably like the clicking of dice tumbling on a wooden table or the ticking of a clock that skips a second.

\textbf{Static Affection.} Touching an Unluck Eater feels pleasant but wrong. Its fur is incredibly soft, yet petting it leaves a lingering static charge on your fingertips. Characters holding an Unluck Eater might find small objects slipping from their grasp or coins disappearing from their pockets, only to reappear in the creature's bedding.

\textbf{The Shadow's Lag.} The most telltale sign of an Unluck Eater is its shadow. It does not move perfectly in sync with the creature. If the Eater jumps, the shadow might wait a fraction of a second before following, or land slightly before the creature does. This is because the Eater exists slightly outside the linear flow of cause and effect.

\subsection*{Tactics \& Behavior}
Unluck Eaters are drawn to individuals of "high destiny"—adventurers, gamblers, and royalty—whose actions cause significant ripples in the fabric of fate. It will bond with a party, acting the part of a loyal mascot.

It instinctively uses its \textit{Glutton for Failure} ability to endear itself to the party, saving them from minor embarrassments or missed attacks. It wants them to succeed, because the higher they climb, the more delicious the eventual fall will be.

When the Eater is "full" (holding many Entropy Charges), it becomes sluggish and heavy. Its eyes may dilate until they are entirely black, and the air around it smells faintly of ozone and copper (the smell of blood).

% Force the Unluck Eater Stat Block to a clean column/page
\clearpage 

\begin{monsterbox}{Unluck Eater}
\textit{Tiny aberration, chaotic neutral}
\statrule
\textbf{Armor Class} 14 (natural armor) \\
\textbf{Hit Points} 21 (6d4 + 6) \\
\textbf{Speed} 30 ft., climb 30 ft.
\statrule
\begin{tabularx}{\linewidth}{XXXXXX}
\textbf{STR} & \textbf{DEX} & \textbf{CON} & \textbf{INT} & \textbf{WIS} & \textbf{CHA} \\
6 (-2) & 16 (+3) & 12 (+1) & 10 (+0) & 14 (+2) & 18 (+4) \\
\end{tabularx}
\statrule
\textbf{Skills} Stealth +5, Persuasion +6 \\
\textbf{Senses} Darkvision 60 ft., passive Perception 12 \\
\textbf{Languages} Understands Common and Deep Speech but can't speak \\
\textbf{Challenge} 1/2 (100 XP)
\statrule

\textit{\textbf{Innocent Guise.}} While the Unluck Eater has not attacked or used its \textit{Entropy Burst}, any creature who targets it with an attack or a harmful spell must first make a \textbf{DC 14 Wisdom saving throw}. On a failed save, the creature must choose a new target or lose the attack or spell.

\textit{\textbf{Entropy Storage.}} The Unluck Eater tracks the number of times it has used \textit{Glutton for Failure}. It has no maximum limit, but if it dies, it explodes. Each creature within 20 feet must make a DC 13 Dexterity saving throw, taking 1d6 force damage per stored charge on a failed save, or half as much on a successful one.

\section*{Actions}
\textit{\textbf{Nip.}} \textit{Melee Weapon Attack:} +5 to hit, reach 5 ft., one target. \textit{Hit:} 1 piercing damage plus 2 (1d4) force damage.

\textit{\textbf{Entropy Burst (Recharge 6).}} The Unluck Eater unleashes a wave of bad luck. It expends up to 5 stored entropy charges. For each charge expended, one creature it can see within 30 feet must succeed on a \textbf{DC 14 Charisma saving throw} or take 2d8 psychic damage and have disadvantage on the next attack roll it makes.

\section*{Reactions}
\textit{\textbf{Glutton for Failure.}} When a creature within 30 feet of the Unluck Eater fails an attack roll, ability check, or saving throw, the Unluck Eater can turn that failure into a \textbf{success}. The Unluck Eater gains 1 entropy charge.
\end{monsterbox}

\vspace{0.5cm}

\begin{sidebarbox}{Synergy: The Gilded Croupier}
If the Unluck Eater is present during a fight with a Gilded Croupier, the Croupier can use its \textbf{Cash Out} legendary action to detonate the Eater, healing itself and damaging players based on the stored Entropy Charges.
\end{sidebarbox}

% --- THE GILDED CROUPIER SECTION ---
\clearpage
\onecolumn      % Switch to Full Width Canvas
\raggedbottom   % Force content to the top

\section*{The Gilded Croupier}

% --- LORE SECTION (Unbalanced Columns: Fill Left then Right) ---
\begin{multicols*}{2}
\textit{"I demanded to speak to the manager. The thing that rolled out from behind the curtain didn't have a face, just a mask of polished mahogany. It told me my credit was declined. Then it stamped my brother, and... he just wasn't there anymore."}
\\—Survivor of the Infinite Casino.

The Proprietor does not demean himself by collecting every petty debt personally. For that, he constructs the Croupiers—imposing, nine-foot-tall constructs of dark mahogany, brass, and green velvet, powered by the rhythmic ticking of an internal probability engine.

\textbf{The House's Enforcer.} A Gilded Croupier presides over the Proprietor’s domains, whether they are literal casinos in the Feywild or soul-vaults in the Shadowfell. They act as impartial judges, ensuring the "games" are played fairly—until it is time for the House to win. To fight a Croupier is to fight an equation that has already solved for your defeat.

% SUBSECTION 1: Physical Description & Mechanics
\subsection*{Constructed Perfection}
\textbf{The Golden Ratio of Violence.} Croupiers are built with four arms to maximize efficiency. Two arms are delicate, tipped with brass fingers for shuffling decks or raking chips with blinding speed. The other two are heavy, piston-driven limbs designed for "customer service." They do not feel pain, pity, or greed.

\textbf{The Ticker Heart.} Inside the chest of every Croupier is a specialized modron-core, ticking in perfect rhythm with the multiverse. A glass window in the center of their torso reveals a constantly scrolling spool of golden parchment—a ticker tape of every debt owed within a 100-mile radius. When a Croupier identifies a debtor, the tape stops, a bell chimes, and the construct initiates "collection protocols."

\textbf{Clockwork Courtesy.} Despite their lethal nature, Croupiers are programmed with a veneer of high-class hospitality. They move with fluid, eerie grace. They speak in pre-recorded, multi-tonal voices, offering polite phrases like \textit{"Please keep your hands off the table"} or \textit{"Your time has expired"} while dismembering intruders.

% SUBSECTION 2: Weapons & Hierarchy
\subsection*{Tools of the Trade}
\textbf{The Void Stamp.} The most feared tool in a Croupier's arsenal is not a weapon of war, but of bureaucracy. The Void Stamp is a heavy brass device wreathed in necrotic magic. When a Croupier deems a creature's debt "unpayable," it does not kill them—it cancels their contract with existence. The stamp leaves behind nothing but a lingering smell of ozone and a memory that is already beginning to fade.

\textbf{Pit Boss Hierarchy.} Croupiers often command squads of lesser constructs, such as Animated Armor painted in livery or tiny Modron runners. In the Proprietor's casinos, the Croupier is the absolute authority; even powerful devils and demons respect the floor rules when a Croupier is watching, for they know that the House has means of collection that transcend immortality.
\end{multicols*}

% --- WIDE STAT BLOCK ---
\begin{monsterbox}{The Gilded Croupier}
\textit{Large construct, lawful neutral}
\statrule
\textbf{Armor Class} 17 (natural armor) \\
\textbf{Hit Points} 114 (12d10 + 48) \\
\textbf{Speed} 30 ft.
\statrule
\begin{tabularx}{\linewidth}{XXXXXX}
\textbf{STR} & \textbf{DEX} & \textbf{CON} & \textbf{INT} & \textbf{WIS} & \textbf{CHA} \\
18 (+4) & 14 (+2) & 18 (+4) & 12 (+1) & 14 (+2) & 1 (-5) \\
\end{tabularx}
\statrule
\textbf{Saving Throws} Con +7, Wis +5 \\
\textbf{Damage Immunities} poison, psychic \\
\textbf{Condition Immunities} charmed, exhaustion, frightened, paralyzed, petrified, poisoned \\
\textbf{Senses} darkvision 60 ft., passive Perception 12 \\
\textbf{Languages} understands Common and Infernal but speaks only in pre-recorded phrases \\
\textbf{Challenge} 7 (2,900 XP)
\statrule

\textit{\textbf{Immutable Form.}} The Croupier is immune to any spell or effect that would alter its form.

\textit{\textbf{Magic Resistance.}} The Croupier has advantage on saving throws against spells and other magical effects.

\textit{\textbf{Calculated Risk.}} The Croupier scores a critical hit on a roll of 19 or 20.

\section*{Actions}
\textit{\textbf{Multiattack.}} The Croupier makes three attacks: two with its Rake and one with its Razor Card.

\textit{\textbf{The House Rake.}} \textit{Melee Weapon Attack:} +7 to hit, reach 10 ft., one target. \textit{Hit:} 13 (2d8 + 4) bludgeoning damage. If the target is a creature, it must succeed on a DC 15 Strength saving throw or be pulled up to 10 feet toward the Croupier and knocked prone.

\textit{\textbf{Razor Card.}} \textit{Ranged Weapon Attack:} +7 to hit, range 30/60 ft., one target. \textit{Hit:} 9 (1d6 + 4) slashing damage plus 7 (2d6) force damage.

\textit{\textbf{Void Stamp (Recharge 5–6).}} The Croupier attempts to cancel a creature's existence. It slams its stamp onto a prone creature within 5 feet. The target must make a \textbf{DC 15 Constitution saving throw}. On a failure, the target takes 33 (6d10) necrotic damage and is stunned until the end of its next turn. On a success, it takes half damage and isn't stunned.

\section*{Legendary Actions}
The Croupier can take 3 legendary actions, choosing from the options below.

\textbf{Deal.} The Croupier makes one Razor Card attack. \\
\textbf{Shuffle (Costs 2 Actions).} The Croupier moves up to its speed without provoking opportunity attacks. \\
\textbf{Cash Out (Costs 3 Actions).} The Croupier targets an Unluck Eater it can see within 60 feet. The Eater is destroyed. All creatures within 20 feet of the Eater must make a DC 15 Dexterity saving throw, taking 1d8 force damage for every \textit{Entropy Charge} the Eater had stored (half on success). The Croupier heals for the same amount.
\end{monsterbox}

% --- RESET FOR REST OF BOOK ---
\twocolumn

\end{document}